\documentclass[a4paper,14pt]{article}
\usepackage[utf8]{vietnam}
\usepackage{fancyhdr}
\usepackage{graphicx,amsfonts,amsmath,amssymb}
\usepackage[top=.75in, left=.75in, right=.75in, bottom=1in]{geometry}
\usepackage{listings}
\usepackage{blindtext}
\usepackage[colorlinks=true,linkcolor=blue, citecolor=red]{hyperref}
\fancyhf{}
\pagestyle{fancy}
% ============ CODE ============
\usepackage{listings}
\usepackage{xcolor}
\definecolor{codegreen}{rgb}{0,0.6,0}
\definecolor{codegray}{rgb}{0.5,0.5,0.5}
\definecolor{codepurple}{rgb}{0.58,0,0.82}
\definecolor{backcolour}{rgb}{0.95,0.95,0.92}

% Styling for the code.
\lstdefinestyle{mystyle}{
	backgroundcolor=\color{backcolour},   
	commentstyle=\color{codegreen},
	keywordstyle=\color{magenta},
	numberstyle=\tiny\color{codegray},
	stringstyle=\color{codepurple},
	basicstyle=\ttfamily\footnotesize,
	breakatwhitespace=false,         
	breaklines=true,                 
	captionpos=b,                    
	keepspaces=true,                 
	numbers=left,                    
	numbersep=5pt,                  
	showspaces=false,                
	showstringspaces=false,
	showtabs=false,                  
	tabsize=2
}
\lstset{style=mystyle}
\begin{document}
	%------------------TẠO TRANG BÌA---------------------------------
	\begin{titlepage}
		\newcommand{\HRule}{\rule{\linewidth}{0.5mm}}
		\centering
		\textsc{\LARGE đại học quốc gia tphcm}\\[0.5cm]
		\textsc{\Large trường đại học khoa học tự nhiên}\\[0.5cm]
		\textsc{\large khoa điện tử viễn thông}\\[0.5cm]
		\HRule \\[0.4cm]
		{
			\huge{\bfseries{Báo cáo Bài tập gì đấy}}\\[0.5cm]
			\large{\bfseries{Đề tài: Tên Báo cáo gì đấy}}
		}\\[0.4cm]
		\HRule \\[0.5cm]
		\textbf{\large Môn học: Môn học gì đấy}\\[0.5cm]
		\begin{minipage}[t]{0.4\textwidth}
			\begin{flushleft} \large
				\emph{Sinh viên thực hiện:}\\
				Sinh viên A\\
				Sinh viên B\\
				Sinh viên C\\
			\end{flushleft}
		\end{minipage}
		\begin{minipage}[t]{0.4\textwidth}
			\begin{flushright} \large
				\begin{flushright}
					\emph{Giáo viên hướng dẫn:}\\
					Tên người hướng dẫn
				\end{flushright}
			\end{flushright}
		\end{minipage}\\[1cm]
		{\large \today}\\[1cm]
		\begin{figure}[h]
			\centering
			\includegraphics[width=0.4\linewidth]{D:/Download/OIP}
			\label{fig:oip}
		\end{figure}
		\vfill
	\end{titlepage}
%------------------------------------------------------------------

%-----------------TẠO HEADER VÀ FOOTER-----------------------------
% Header length
\setlength{\headheight}{29.43912pt}
% Header
\pagestyle{fancy}
\lhead{Báo cáo Bài tập gì đấy}
\rhead{Trường Đại học Khoa học Tự nhiên - ĐHQG HCM\\
	Môn học gì đấy}
% Footer
\newcommand{\leftfooter}{\textbf{\LaTeX}\ by \href{https://github.com/Nam000p}{Đặng Xuân Nam}}
\lfoot{\leftfooter}
\fancyfoot[R]{Trang \thepage \hspace{1pt}}
%------------------------------------------------------------------

%-------TẠO MỤC LỤC, DANH SÁC BẢNG VÀ DANH SÁCH HÌNH VẼ------------
\tableofcontents
\pagebreak
\listoftables
\pagebreak
\listoffigures
\pagebreak
%------------------------------------------------------------------

%------------SECTION 1--------------------------------------------
	\section{Section}
	
	\subsection{Một số lưu ý}
	
	\subsubsection{Cài đặt offline}
	Template này yêu cầu cài đặt một số gói (package) nâng cao cho TexStudio:
	\begin{itemize}
		\item Để gõ thuật toán: \texttt{algorithm} và \texttt{algpseudocode}
		\item Để nhúng (chèn) code: \texttt{listings}
	\end{itemize}
	Các gói này được cài đặt thông qua lệnh
	\begin{lstlisting}[language=sh]
		sudo apt-get install texlive-full
	\end{lstlisting}
	Tuy nhiên kích thước gói đâu đó vào khoảng 5GB (!). Vì vậy tốt nhất nên xài Overleaf.
	
	\subsubsection{Sử dụng font khác}
	Tham khảo font typefaces tại \href{https://www.overleaf.com/learn/latex/Font_typefaces}{link này}.
	
	\subsubsection{Đánh số chỉ mục bằng chữ số La Mã}
	Mở file \texttt{main.tex} và bỏ comment dòng 
	\begin{lstlisting}[language=tex]
		% \renewcommand{\thesection}{\Roman{section}}
		% \renewcommand{\thesubsection}{\thesection.\Roman{subsection}}  
	\end{lstlisting}
	
	\subsection{Ví dụ}
	Ngày xửa ngày xưa, ở vương quốc VNUHCM - US, có một chàng hoàng tử ngồi cắm đầu viết doc\footnote{Đây là footnote, chú thích lại những gì cần chú ý.}.\\
	Mặc định muốn xuống dòng chỉ cần dùng $\backslash\backslash$  (2 lần dấu xẹt huyền).\\
	Nếu bạn muốn thụt đầu dòng khi bắt đầu paragraph mới, vào \texttt{main.tex} và disble dòng
	\begin{lstlisting}[language=tex]
		\setlength{\parindent}{0pt}
	\end{lstlisting}

	\subsection{First subsection}
	\subsubsection{First sub-subsection}
	Subsection để ví dụ thôi. Thêm vài ví dụ:
	\begin{itemize}
		\item Dùng itemize
		\item Vẫn là itemize
	\end{itemize}
	Sau đó xài enumerate:
	\begin{enumerate}
		\item Dùng enumerate
		\item Vẫn là enumerate
	\end{enumerate}
	Nhỏ hơn subsubsection thì xài \texttt{paragraph}:
	
	\paragraph{Đây là ví dụ cho paragraph}
	Lưu ý là paragraph không nằm trong Mục lục.
	
	\subsection{Chia nhỏ nội dung}
	Bạn có thể chia nhỏ nội dung của báo cáo thành các file \texttt{.tex} và dùng lệnh \texttt{input} để chèn vào báo cáo chính. Ví dụ có trong file \texttt{main.tex}.
%-------------------------------------------------------------------------------------

%----------------------SECTION 2------------------------------------------------------
	\section{Hình ảnh}
	Hình ảnh được thể hiện như hình~\ref{fig:oip}, lưu ý flag \texttt{[H]} để disable floating (hình được hiển thị đúng vị trí, không trôi lên đầu trang).
	\begin{figure}%[h]
		\centering
		\includegraphics[scale=.4]{D:/Download/OIP}
		\caption{Hình ví dụ (logo HCMUS - updated 30/11/2022)}
		\label{fig:oip}
	\end{figure}
	Hình~\ref{fig:oip_with_h} cũng là hình ví dụ nhưng có tag \texttt{[H]}. Lưu ý là có tag \texttt{[H]} thì code ở đâu hình sẽ nằm ở đó, không quan trọng nội dung ít hay nhiều (trang giấy sẽ thừa 1 khúc như bạn thấy). Để hiểu hơn về positioning trong LaTeX, xin tham khảo.
	\begin{figure}[h]
		\centering
		\includegraphics[scale=.4]{D:/Download/OIP}
		\caption{Hình ví dụ (logo HCMUS - updated 30/11/2022)}
		\label{fig:oip_with_h}
	\end{figure}
%--------------------------------------------------------------------------------

%--------------------------SECTION 3---------------------------------------------
	\section{Bảng biểu}
	Bảng biểu được thể hiện như bảng~\ref{tab:my_label}, lưu ý flag \texttt{[H]} để disable floating (bảng được hiển thị đúng vị trí, không trôi lên đầu trang). Bảng~\ref{tab:my_label} là một trường hợp không sử dụng tag \texttt{[H]} và bảng bị trôi tít lên đầu trang:
	\begin{table}%[H]
		\centering
		\begin{tabular}{|l|l|}
			\hline
			\textbf{Tên con vật} & \textbf{Số chân} \\ \hline
			Gà & 2 \\ \hline
			Chó & 4 \\ \hline
			Trần Hoàng Tử & 2 \\ \hline
		\end{tabular}
		\caption{Số chân của một số con vật, không có tag \texttt{[H]}}
		\label{tab:my_label}
	\end{table}
	Bảng~\ref{tab:my_label_with_H_tag} thể hiện bảng biểu với tag \texttt{[H]}\footnote{Tương tự cách sử dụng tag \texttt{[H]} với hình}. Để không phải mất thời gian tuổi trẻ ngồi chỉnh table, xài.
	\begin{table}[h]
		\centering
		\begin{tabular}{|l|l|}
			\hline
			\textbf{Tên con vật} & \textbf{Số chân} \\ \hline
			Gà & 2 \\ \hline
			Chó & 4 \\ \hline
			Trần Hoàng Tử & 2 \\ \hline
		\end{tabular}
		\caption{Số chân của một số con vật, có tag \texttt{[H]}}
		\label{tab:my_label_with_H_tag}
	\end{table}.
%-----------------------------------------------------------------------

%----------------------SECTION 4----------------------------------------
\section{Công thức toán}
Công thức toán gõ chung 1 dòng thì dùng 2 lần dấu dollar: $f(x) = x^2 + 2x + 1$. Với công thức nằm riêng 1 dòng thì gõ 2 cặp dấu dollar:
$$
ReLU(x) = \max(0, x)
$$
Siêu việt hơn, gõ hệ phương trình thì nên dùng tag \texttt{equation}
\begin{equation*}
	\begin{aligned}
		a_1x_1 + a_2x_2 + .. + a_nx_n &= u \\
		b_1x_1 + b_2x_2 + .. + b_nx_n &= v \\
		c_1x_1 + c_2x_2 + .. + c_nx_n &= w \\
	\end{aligned}
\end{equation*}
Tham khảo cách gõ equation trên \href{https://www.overleaf.com/learn/latex/Mathematical_expressions}{Overleaf} nhé!
%----------------------------------------------------------------------

%-------------------SECTION 5------------------------------------------
\section{Ngôn ngữ}
Ngôn ngữ mặc định của template là Tiếng Việt, config ở file \texttt{main.tex} với lệnh
\begin{lstlisting}[language=tex]
	\usepackage[utf8]{vietnam}
\end{lstlisting}
Để chuyển sang Tiếng Anh (e.g. nhiều khi bạn muốn label trong các bảng bằng Tiếng Anh; bạn muốn viết report bằng Tiếng Anh thay vì Tiếng Việt), khi đó có 2 lựa chọn:
\begin{itemize}
	\item Chuyển xang xài package \texttt{babel} và xài tag \texttt{$\backslash$uselanguage}.
	\item Bỏ xài package \texttt{vietnam}
\end{itemize}
Hướng dẫn thì mời bạn xem \href{https://www.overleaf.com/learn/latex/International_language_support#Babel}{link này}
%---------------------------------------------------------------------

%------------------------SECTION 6------------------------------------
\section{Code}
Dùng gói \texttt{listings} để gõ code, ví dụ cho C++:
\begin{lstlisting}[language=C++]
#include <iostream>
int main(){
	std::cout << "Hello World!"\n;
	return 0;
}
\end{lstlisting}

Cho Python:
\begin{lstlisting}[label=Python]
	print("Hello World!")
\end{lstlisting}

%---------------------------------------------------------------------
%------------------------SECTION 7------------------------------------
\section{Sử dụng tài liệu tham khảo}

File BibTeX tài liệu tham khảo nằm ở đường dẫn \texttt{ref/ref.bib}. Sửa tên file \texttt{.bib} sẽ phải sửa lại nội dung file \texttt{ref.tex}.

Đây là ví dụ cite một tài liệu\cite{texbook}\cite{lamport94}.

%---------------------------------------------------------------------

%---
\cleardoublepage
\phantomsection
\addcontentsline{toc}{section}{Tài liệu}
\bibliographystyle{plain}
\begin{thebibliography}{9}
	\bibitem{texbook}
	Donald E. Knuth (1986) \emph{The \TeX{} Book}, Addison-Wesley Professional.
	
	\bibitem{lamport94}
	Leslie Lamport (1994) \emph{\LaTeX: a document preparation system}, Addison
	Wesley, Massachusetts, 2nd ed.
\end{thebibliography}
\end{document}